%%%%%%%%%%%%%%%%%%%%%%%%%%%%%%%%
% Chap 5. Conclusion and Future Work
%%%%%%%%%%%%%%%%%%%%%%%%%%%%%%%%

\chapter{Conclusion and Future Work} \label{chapter5}

In this thesis, the Constrained Optimization-based Neuro-Adaptive Controller (Co-\allowbreak NAC) for uncertain Euler-Lagrange systems is presented. 
The two simulation validations showed that the CoNAC can satisfy the imposed constraints regarding boundedness of neural network's weights and control input saturation, while achieving the desired tracking performance.

However, neuro-adaptive control (NAC) methods including CoNAC have several limitations to be referred as deep learning-based controller.
First, NAC methods adapts their weights to reduce objective function using current tracking error. 
This means online implementation is required to train the NNs since the tracking error is dependent on the current NNs' weights.
Moreover, for the same reason, the NNs cannot be trained offline.
Second, the gradient vanishing problem still exists in the train process of the NNs.
To overcome this issue, simply ReLU activation function can be used.
However, the stability should be examined rigorously, since this activation function is unbounded.

The following future works are suggested to tackle above limitations.
First, reinforcement learning (RL) approach can be used, since training NNs with respect to the tracking error is similar to maximizing the reward in RL.
There are some literature whose basis is optimal control theory \cite{RN119,RN120,RN121}.
They approximate optimal control law which is ideally obtained using Hamilton-Jacobi-Bellman framework.
Second, once the limitation that the offline adaptation is unavailable is solved, stochastic problem formulation can be used to theoretically utilize novel deep learning methods.
Since the recent deep learning methods are based on stochastic methods (\eg, stochastic gradient descent (SGD), drop out, $L_2$-regularization), stochastic stability analysis should be conducted.
There are some attempts to analyze the stability of system which uses stochastic neural network \cite{RN122,RN123}.
Third, other novel constrained optimization methods can be used to solve gradient vanishing issue.
Except gradient descent-like methods, the existing methods to overcome gradient vanishing issue using constrained optimization approach such as the augmented Lagrangian approach \cite{RN62} and the alternating direction method of multipliers \cite{RN98, RN94} are introduced.
These methods transform the layer's architecture of the NNs to equality constraints and optimize each layer's weights and output of activation functions.

